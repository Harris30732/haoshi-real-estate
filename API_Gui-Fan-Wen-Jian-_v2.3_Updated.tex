\documentclass[12pt, a4paper]{article}

\usepackage[utf8]{inputenc}
\usepackage[margin=2cm]{geometry}
\usepackage{xcolor}
\usepackage{listings}
\usepackage{hyperref}
\usepackage{array}
\usepackage{xeCJK}
\usepackage{fancyhdr}
\usepackage{lastpage}
\usepackage{tabularx}

\setCJKmainfont{SimSun}

\lstset{
basicstyle=\ttfamily\small,
breaklines=true,
columns=fullflexible,
frame=single,
backgroundcolor=\color{gray!10},
keywordstyle=\color{blue},
commentstyle=\color{green!60!black},
stringstyle=\color{red}
}

\pagestyle{fancy}
\lhead{房屋管理系統 API 規範}
\rhead{第 \thepage\ 頁,共 \pageref{LastPage}\ 頁}
\cfoot{\small 更新時間:2026-01-06}

\title{\textbf{房屋管理系統 API 規範文件}}
\author{API 文件版本 2.3}
\date{2026年1月6日}

\begin{document}

\maketitle

\tableofcontents

\newpage

\section{簡介}

本文件詳細說明房屋管理系統的核心 API 端點,涵蓋完整的三層權限架構。

\subsection{版本歷史}

\begin{tabularx}{\textwidth}{|l|l|X|}
\hline
\textbf{版本} & \textbf{日期} & \textbf{更新內容} \\
\hline
1.0 & 2025-12-31 & 初始版本:資料查詢、CRUD 操作、照片上傳 \\
\hline
2.0 & 2025-12-31 & 新增 Google OAuth 登入 API、JWT 認證系統 \\
\hline
2.1 & 2026-01-01 & 新增使用者管理 API(CRUD) \\
\hline
2.2 & 2026-01-01 & 重新設計三層權限架構(user, manager, admin) \\
\hline
2.3 & 2026-01-06 & 更新實際 API 路徑,補充完整 INPUT 規格 \\
\hline
\end{tabularx}

\subsection{技術架構}

\begin{itemize}
\item \textbf{後端框架}:n8n Workflow Automation
\item \textbf{資料庫}:Supabase PostgreSQL
\item \textbf{認證方式}:Google OAuth 2.0 + JWT Token
\item \textbf{權限系統}:三層角色權限(user, manager, admin)
\item \textbf{檔案儲存}:Supabase Storage
\item \textbf{API 基礎網址}:https://findmyhome.zeabur.app/webhook
\item \textbf{測試環境}:https://findmyhome.zeabur.app/webhook-test
\end{itemize}

\newpage

\section{權限系統說明}

\subsection{三層角色定義}

\begin{tabularx}{\textwidth}{|l|l|X|}
\hline
\textbf{角色} & \textbf{英文代碼} & \textbf{權限說明} \\
\hline
一般使用者 & user & 僅能查看公開資料、修改自己的個人資料 \\
\hline
管理者 & manager & 可管理房屋物件、社區資料,可查看使用者列表 \\
\hline
系統管理員 & admin & 完整權限,可管理所有功能包括使用者帳號 \\
\hline
\end{tabularx}

\subsection{完整權限矩陣}

\begin{tabularx}{\textwidth}{|l|c|c|c|X|}
\hline
\textbf{功能} & \textbf{User} & \textbf{Manager} & \textbf{Admin} & \textbf{說明} \\
\hline
\multicolumn{5}{|l|}{\textbf{公開資料}} \\
\hline
查看所有物件 & ✓ & ✓ & ✓ & 所有人可查看 \\
\hline
查看所有社區 & ✓ & ✓ & ✓ & 所有人可查看 \\
\hline
查看使用者列表(基本資訊) & ✓ & ✓ & ✓ & 所有人可查看 \\
\hline
\multicolumn{5}{|l|}{\textbf{個人資料}} \\
\hline
查看自己的完整資料 & ✓ & ✓ & ✓ & 可查看自己 \\
\hline
修改自己的個人資料 & ✓ & ✓ & ✓ & 可修改自己 \\
\hline
\multicolumn{5}{|l|}{\textbf{房屋物件管理}} \\
\hline
新增物件 & ✗ & ✓ & ✓ & Manager 以上 \\
\hline
修改物件 & ✗ & ✓ & ✓ & Manager 以上 \\
\hline
刪除物件 & ✗ & ✓ & ✓ & Manager 以上 \\
\hline
上傳物件照片 & ✗ & ✓ & ✓ & Manager 以上 \\
\hline
\multicolumn{5}{|l|}{\textbf{社區管理}} \\
\hline
新增社區 & ✗ & ✓ & ✓ & Manager 以上 \\
\hline
修改社區 & ✗ & ✓ & ✓ & Manager 以上 \\
\hline
刪除社區 & ✗ & ✓ & ✓ & Manager 以上 \\
\hline
\multicolumn{5}{|l|}{\textbf{使用者管理}} \\
\hline
查看所有使用者(完整資訊) & ✗ & ✗ & ✓ & 僅 Admin \\
\hline
新增使用者 & ✗ & ✗ & ✓ & 僅 Admin \\
\hline
修改他人資料 & ✗ & ✗ & ✓ & 僅 Admin \\
\hline
更新使用者角色 & ✗ & ✗ & ✓ & 僅 Admin \\
\hline
刪除使用者 & ✗ & ✗ & ✓ & 僅 Admin \\
\hline
\end{tabularx}

\newpage

\section{API 1:使用者管理}

\subsection{基本資訊}

\begin{tabularx}{\textwidth}{|l|X|}
\hline
\textbf{項目} & \textbf{說明} \\
\hline
功能說明 & 管理使用者資料與角色(新增、修改、刪除) \\
\hline
請求方法 & POST \\
\hline
API 路徑 & /webhook/admin/users \\
\hline
需要認證 & 是(僅 admin 可執行) \\
\hline
權限要求 & admin \\
\hline
\end{tabularx}

\subsection{操作 1:新增使用者(Create)}

\subsubsection{請求參數}

\begin{lstlisting}[language=JavaScript]
{
  "action": "create",
  "user": "管理員",
  "data": {
    "email": "manager@example.com",
    "username": "manager",
    "name": "王經理",
    "display_name": "王大明",
    "title": "資深業務經理",
    "role": "user"
  }
}
\end{lstlisting}

\subsubsection{參數說明}

\begin{tabularx}{\textwidth}{|l|l|l|X|}
\hline
\textbf{欄位} & \textbf{必填} & \textbf{類型} & \textbf{說明} \\
\hline
action & ✓ & string & 固定值:create \\
\hline
user & ✓ & string & 操作者名稱 \\
\hline
data.email & ✓ & string & 使用者電子郵件 \\
\hline
data.username & ✓ & string & 使用者帳號 \\
\hline
data.name & ✓ & string & 使用者姓名 \\
\hline
data.display\_name & & string & 顯示名稱 \\
\hline
data.title & & string & 職稱 \\
\hline
data.role & ✓ & string & 角色(user/manager/admin) \\
\hline
\end{tabularx}

\subsubsection{成功回應}

\begin{lstlisting}[language=JavaScript]
{
  "status": "success",
  "action": "create",
  "message": "使用者新增成功",
  "data": {
    "id": "新生成的UUID",
    "email": "manager@example.com",
    "username": "manager",
    "name": "王經理",
    "display_name": "王大明",
    "title": "資深業務經理",
    "role": "user",
    "created_at": "2026-01-06T16:34:00Z",
    "created_by": "管理員"
  }
}
\end{lstlisting}

\subsection{操作 2:修改使用者資料(Update)}

\subsubsection{請求參數}

\begin{lstlisting}[language=JavaScript]
{
  "action": "update",
  "user": "系統管理員",
  "id": "cff46d5a-8d50-48a0-b9b3-92bdf0dde378",
  "data": {
    "email": "manager@example.com",
    "name": "王經理",
    "title": "資深業務經理",
    "role": "admin"
  }
}
\end{lstlisting}

\subsubsection{參數說明}

\begin{tabularx}{\textwidth}{|l|l|l|X|}
\hline
\textbf{欄位} & \textbf{必填} & \textbf{類型} & \textbf{說明} \\
\hline
action & ✓ & string & 固定值:update \\
\hline
user & ✓ & string & 操作者名稱 \\
\hline
id & ✓ & string & 要修改的使用者 UUID \\
\hline
data & ✓ & object & 要更新的欄位(僅需包含需修改的欄位) \\
\hline
\end{tabularx}

\textbf{注意事項:}
\begin{itemize}
\item data 中僅需包含需要更新的欄位
\item 未提供的欄位將保持原值不變
\item 回傳時會返回完整的使用者物件
\end{itemize}

\subsubsection{成功回應}

\begin{lstlisting}[language=JavaScript]
{
  "status": "success",
  "action": "update",
  "message": "使用者資料已更新",
  "data": {
    "id": "cff46d5a-8d50-48a0-b9b3-92bdf0dde378",
    "email": "manager@example.com",
    "username": "manager",
    "name": "王經理",
    "display_name": "王大明",
    "title": "資深業務經理",
    "role": "admin",
    "updated_at": "2026-01-06T16:34:00Z",
    "updated_by": "系統管理員"
  }
}
\end{lstlisting}

\newpage

\section{API 2:社區管理}

\subsection{基本資訊}

\begin{tabularx}{\textwidth}{|l|X|}
\hline
\textbf{項目} & \textbf{說明} \\
\hline
功能說明 & 管理社區資料(新增、修改、刪除) \\
\hline
請求方法 & POST \\
\hline
API 路徑(正式)& /webhook/admin/communities \\
\hline
API 路徑(測試)& /webhook-test/admin/communities \\
\hline
需要認證 & 是 \\
\hline
權限要求 & manager 或 admin \\
\hline
\end{tabularx}

\subsection{操作 1:新增社區(Create)}

\subsubsection{請求參數}

\begin{lstlisting}[language=JavaScript]
{
  "action": "create",
  "user": "錦宣",
  "data": {
    "builder": "測試建商",
    "community_name": "【測試用】待刪除社區",
    "completion_date": "110",
    "total_units": "99",
    "unit_area_range": "30/40"
  }
}
\end{lstlisting}

\subsubsection{參數說明}

\begin{tabularx}{\textwidth}{|l|l|l|X|}
\hline
\textbf{欄位} & \textbf{必填} & \textbf{類型} & \textbf{說明} \\
\hline
action & ✓ & string & 固定值:create \\
\hline
user & ✓ & string & 操作者名稱 \\
\hline
data.builder & ✓ & string & 建商名稱 \\
\hline
data.community\_name & ✓ & string & 社區名稱 \\
\hline
data.completion\_date & ✓ & string & 完工年份(民國年) \\
\hline
data.total\_units & ✓ & string & 總戶數 \\
\hline
data.unit\_area\_range & ✓ & string & 坪數範圍(格式:最小/最大) \\
\hline
\end{tabularx}

\subsubsection{成功回應}

\begin{lstlisting}[language=JavaScript]
{
  "status": "success",
  "action": "create",
  "message": "社區新增成功",
  "data": {
    "id": "新生成的UUID",
    "builder": "測試建商",
    "community_name": "【測試用】待刪除社區",
    "completion_date": "110",
    "total_units": "99",
    "unit_area_range": "30/40",
    "created_at": "2026-01-06T16:34:00Z",
    "created_by": "錦宣"
  }
}
\end{lstlisting}

\subsection{操作 2:修改社區(Update)}

\subsubsection{請求參數}

\begin{lstlisting}[language=JavaScript]
{
  "action": "update",
  "user": "錦宣",
  "id": "beb6a14c-dbb0-407f-a29f-5e3067ab9758",
  "data": {
    "total_units": "120",
    "completion_date": "111"
  }
}
\end{lstlisting}

\subsubsection{參數說明}

\begin{tabularx}{\textwidth}{|l|l|l|X|}
\hline
\textbf{欄位} & \textbf{必填} & \textbf{類型} & \textbf{說明} \\
\hline
action & ✓ & string & 固定值:update \\
\hline
user & ✓ & string & 操作者名稱 \\
\hline
id & ✓ & string & 要修改的社區 UUID \\
\hline
data & ✓ & object & 要更新的欄位(僅需包含需修改的欄位) \\
\hline
\end{tabularx}

\textbf{注意事項:}
\begin{itemize}
\item data 中僅需包含需要更新的欄位
\item 未提供的欄位將保持原值不變
\item 回傳時會返回完整的社區物件
\end{itemize}

\subsection{操作 3:刪除社區(Delete)}

\subsubsection{請求參數}

\begin{lstlisting}[language=JavaScript]
{
  "action": "delete",
  "user": "錦宣",
  "id": "beb6a14c-dbb0-407f-a29f-5e3067ab9758"
}
\end{lstlisting}

\subsubsection{成功回應}

\begin{lstlisting}[language=JavaScript]
{
  "status": "success",
  "action": "delete",
  "message": "社區已刪除",
  "deleted_id": "beb6a14c-dbb0-407f-a29f-5e3067ab9758"
}
\end{lstlisting}

\newpage

\section{API 3:物件管理}

\subsection{基本資訊}

\begin{tabularx}{\textwidth}{|l|X|}
\hline
\textbf{項目} & \textbf{說明} \\
\hline
功能說明 & 管理房屋物件資料(新增、修改、刪除) \\
\hline
請求方法 & POST \\
\hline
API 路徑(正式)& /webhook/admin/properties \\
\hline
API 路徑(測試)& /webhook-test/admin/properties \\
\hline
需要認證 & 是 \\
\hline
權限要求 & manager 或 admin \\
\hline
\end{tabularx}

\subsection{操作 1:新增物件(Create)}

\subsubsection{請求參數}

\begin{lstlisting}[language=JavaScript]
{
  "action": "create",
  "user": "錦宣",
  "data": {
    "community_name": "遠雄文華匯",
    "total_price": 1200,
    "total_ping": 30.5,
    "parking_ping": 5.2,
    "parking_price": 200,
    "floor_info": "8樓/共15樓",
    "address": "桃園市中壢區中央西路123號8樓",
    "status": "專任",
    "layout": "3房2廳2衛",
    "notes": "面公園,採光佳,屋況良好"
  }
}
\end{lstlisting}

\subsubsection{參數說明}

\begin{tabularx}{\textwidth}{|l|l|l|X|}
\hline
\textbf{欄位} & \textbf{必填} & \textbf{類型} & \textbf{說明} \\
\hline
action & ✓ & string & 固定值:create \\
\hline
user & ✓ & string & 操作者名稱 \\
\hline
data.community\_name & ✓ & string & 社區名稱 \\
\hline
data.total\_price & ✓ & number & 總價(單位:萬) \\
\hline
data.total\_ping & ✓ & number & 總坪數 \\
\hline
data.parking\_ping & & number & 車位坪數 \\
\hline
data.parking\_price & & number & 車位價格(單位:萬) \\
\hline
data.floor\_info & ✓ & string & 樓層資訊 \\
\hline
data.address & ✓ & string & 地址 \\
\hline
data.status & ✓ & string & 狀態(專任/一般/已成交等) \\
\hline
data.layout & ✓ & string & 格局(如:3房2廳2衛) \\
\hline
data.notes & & string & 備註 \\
\hline
\end{tabularx}

\subsubsection{成功回應}

\begin{lstlisting}[language=JavaScript]
{
  "status": "success",
  "action": "create",
  "message": "物件新增成功",
  "data": {
    "id": "067cc432-7dca-4898-a25c-c08065e480ea",
    "community_name": "遠雄文華匯",
    "total_price": 1200,
    "total_ping": 30.5,
    "parking_ping": 5.2,
    "parking_price": 200,
    "floor_info": "8樓/共15樓",
    "address": "桃園市中壢區中央西路123號8樓",
    "status": "專任",
    "layout": "3房2廳2衛",
    "notes": "面公園,採光佳,屋況良好",
    "created_at": "2026-01-06T16:34:00Z",
    "created_by": "錦宣"
  }
}
\end{lstlisting}

\subsection{操作 2:修改物件(Update)}

\subsubsection{請求參數}

\begin{lstlisting}[language=JavaScript]
{
  "action": "update",
  "user": "錦宣",
  "id": "067cc432-7dca-4898-a25c-c08065e480ea",
  "data": {
    "total_price": 1280,
    "status": "已成交"
  }
}
\end{lstlisting}

\subsubsection{參數說明}

\begin{tabularx}{\textwidth}{|l|l|l|X|}
\hline
\textbf{欄位} & \textbf{必填} & \textbf{類型} & \textbf{說明} \\
\hline
action & ✓ & string & 固定值:update \\
\hline
user & ✓ & string & 操作者名稱 \\
\hline
id & ✓ & string & 要修改的物件 UUID \\
\hline
data & ✓ & object & 要更新的欄位(僅需包含需修改的欄位) \\
\hline
\end{tabularx}

\textbf{注意事項:}
\begin{itemize}
\item data 中僅需包含需要更新的欄位
\item 未提供的欄位將保持原值不變
\item 回傳時會返回完整的物件內容
\end{itemize}

\subsubsection{成功回應}

\begin{lstlisting}[language=JavaScript]
{
  "status": "success",
  "action": "update",
  "message": "物件資料已更新",
  "data": {
    "id": "067cc432-7dca-4898-a25c-c08065e480ea",
    "community_name": "遠雄文華匯",
    "total_price": 1280,
    "total_ping": 30.5,
    "parking_ping": 5.2,
    "parking_price": 200,
    "floor_info": "8樓/共15樓",
    "address": "桃園市中壢區中央西路123號8樓",
    "status": "已成交",
    "layout": "3房2廳2衛",
    "notes": "面公園,採光佳,屋況良好",
    "updated_at": "2026-01-06T16:34:00Z",
    "updated_by": "錦宣"
  }
}
\end{lstlisting}

\subsection{操作 3:刪除物件(Delete)}

\subsubsection{請求參數}

\begin{lstlisting}[language=JavaScript]
{
  "action": "delete",
  "user": "錦宣",
  "id": "067cc432-7dca-4898-a25c-c08065e480ea"
}
\end{lstlisting}

\subsubsection{成功回應}

\begin{lstlisting}[language=JavaScript]
{
  "status": "success",
  "action": "delete",
  "message": "物件已刪除",
  "deleted_id": "067cc432-7dca-4898-a25c-c08065e480ea"
}
\end{lstlisting}

\newpage

\section{API 4:照片上傳}

\subsection{基本資訊}

\begin{tabularx}{\textwidth}{|l|X|}
\hline
\textbf{項目} & \textbf{說明} \\
\hline
功能說明 & 上傳物件照片至 Supabase Storage \\
\hline
請求方法 & POST \\
\hline
API 路徑 & /webhook/admin/photos/upload \\
\hline
Content-Type & multipart/form-data \\
\hline
需要認證 & 是 \\
\hline
權限要求 & manager 或 admin \\
\hline
\end{tabularx}

\subsection{請求參數(multipart/form-data)}

\begin{tabularx}{\textwidth}{|l|l|l|X|}
\hline
\textbf{欄位名稱} & \textbf{必填} & \textbf{類型} & \textbf{說明} \\
\hline
property\_id & ✓ & string & 物件 UUID \\
\hline
file & ✓ & file & 圖片檔案(支援 jpg, jpeg, png) \\
\hline
user & ✓ & string & 操作者名稱 \\
\hline
description & & string & 照片描述 \\
\hline
\end{tabularx}

\subsection{INPUT 規格說明}

\textbf{檔案規格:}
\begin{itemize}
\item \textbf{支援格式}:JPG, JPEG, PNG
\item \textbf{檔案大小限制}:最大 10 MB
\item \textbf{建議尺寸}:1920x1080 或更小
\item \textbf{編碼方式}:multipart/form-data
\end{itemize}

\textbf{請求範例(使用 curl):}

\begin{lstlisting}[language=bash]
curl -X POST \
  https://findmyhome.zeabur.app/webhook/admin/photos/upload \
  -H "Authorization: Bearer YOUR_JWT_TOKEN" \
  -F "property_id=067cc432-7dca-4898-a25c-c08065e480ea" \
  -F "user=錦宣" \
  -F "file=@/path/to/image.jpg" \
  -F "description=客廳照片"
\end{lstlisting}

\textbf{請求範例(JavaScript FormData):}

\begin{lstlisting}[language=JavaScript]
const formData = new FormData();
formData.append('property_id', '067cc432-7dca-4898-a25c-c08065e480ea');
formData.append('user', '錦宣');
formData.append('file', fileInput.files[0]); // HTML input element
formData.append('description', '客廳照片');

fetch('https://findmyhome.zeabur.app/webhook/admin/photos/upload', {
  method: 'POST',
  headers: {
    'Authorization': `Bearer ${jwtToken}`
  },
  body: formData
})
.then(response => response.json())
.then(data => console.log(data));
\end{lstlisting}

\subsection{成功回應}

\begin{lstlisting}[language=JavaScript]
{
  "status": "success",
  "message": "照片上傳成功",
  "data": {
    "photo_id": "新生成的UUID",
    "property_id": "067cc432-7dca-4898-a25c-c08065e480ea",
    "url": "https://supabase-storage-url/photos/filename.jpg",
    "description": "客廳照片",
    "uploaded_at": "2026-01-06T16:34:00Z",
    "uploaded_by": "錦宣"
  }
}
\end{lstlisting}

\subsection{錯誤回應}

\begin{lstlisting}[language=JavaScript]
// 檔案格式錯誤
{
  "status": "error",
  "code": "INVALID_FILE_FORMAT",
  "message": "不支援的檔案格式,僅接受 JPG, JPEG, PNG"
}

// 檔案過大
{
  "status": "error",
  "code": "FILE_TOO_LARGE",
  "message": "檔案大小超過限制(最大 10 MB)"
}

// 缺少必填欄位
{
  "status": "error",
  "code": "MISSING_REQUIRED_FIELD",
  "message": "缺少必填欄位:property_id"
}
\end{lstlisting}

\newpage

\section{錯誤代碼與處理}

\subsection{認證與權限錯誤}

\begin{tabularx}{\textwidth}{|l|l|X|}
\hline
\textbf{錯誤代碼} & \textbf{HTTP} & \textbf{說明} \\
\hline
INVALID\_TOKEN & 401 & Google ID Token 或 JWT 無效 \\
\hline
JWT\_EXPIRED & 401 & JWT Token 已過期 \\
\hline
MISSING\_TOKEN & 401 & 缺少認證 Token \\
\hline
INSUFFICIENT\_PERMISSIONS & 403 & 權限不足 \\
\hline
ROLE\_REQUIRED\_MANAGER & 403 & 需要 manager 或更高權限 \\
\hline
ROLE\_REQUIRED\_ADMIN & 403 & 需要 admin 權限 \\
\hline
\end{tabularx}

\subsection{資料驗證錯誤}

\begin{tabularx}{\textwidth}{|l|l|X|}
\hline
\textbf{錯誤代碼} & \textbf{HTTP} & \textbf{說明} \\
\hline
MISSING\_REQUIRED\_FIELD & 400 & 缺少必填欄位 \\
\hline
INVALID\_ACTION & 400 & 無效的操作類型 \\
\hline
INVALID\_UUID & 400 & 無效的 UUID 格式 \\
\hline
RECORD\_NOT\_FOUND & 404 & 找不到指定記錄 \\
\hline
DUPLICATE\_ENTRY & 400 & 資料重複(如 email 已存在) \\
\hline
\end{tabularx}

\subsection{檔案上傳錯誤}

\begin{tabularx}{\textwidth}{|l|l|X|}
\hline
\textbf{錯誤代碼} & \textbf{HTTP} & \textbf{說明} \\
\hline
INVALID\_FILE\_FORMAT & 400 & 不支援的檔案格式 \\
\hline
FILE\_TOO\_LARGE & 400 & 檔案大小超過限制 \\
\hline
UPLOAD\_FAILED & 500 & 檔案上傳失敗 \\
\hline
\end{tabularx}

\newpage

\section{前端整合範例}

\subsection{權限檢查 Hook(React)}

\begin{lstlisting}[language=JavaScript]
import { useState, useEffect } from 'react';
import { jwtDecode } from 'jwt-decode';

function useAuth() {
  const [user, setUser] = useState(null);
  const [role, setRole] = useState(null);

  useEffect(() => {
    const token = localStorage.getItem('access_token');
    if (token) {
      try {
        const decoded = jwtDecode(token);
        setUser(decoded);
        setRole(decoded.role);
      } catch (error) {
        console.error('Invalid token');
        localStorage.removeItem('access_token');
      }
    }
  }, []);

  const hasRole = (requiredRole) => {
    if (!role) return false;
    const roleHierarchy = { user: 0, manager: 1, admin: 2 };
    return roleHierarchy[role] >= roleHierarchy[requiredRole];
  };

  const canManageData = () => hasRole('manager');
  const canManageUsers = () => role === 'admin';

  return {
    user,
    role,
    hasRole,
    canManageData,
    canManageUsers
  };
}

export default useAuth;
\end{lstlisting}

\subsection{API 呼叫範例}

\begin{lstlisting}[language=JavaScript]
// 新增社區
async function createCommunity(communityData) {
  const token = localStorage.getItem('access_token');

  const response = await fetch(
    'https://findmyhome.zeabur.app/webhook/admin/communities',
    {
      method: 'POST',
      headers: {
        'Content-Type': 'application/json',
        'Authorization': `Bearer ${token}`
      },
      body: JSON.stringify({
        action: 'create',
        user: '錦宣',
        data: communityData
      })
    }
  );

  return await response.json();
}

// 修改物件
async function updateProperty(propertyId, updates) {
  const token = localStorage.getItem('access_token');

  const response = await fetch(
    'https://findmyhome.zeabur.app/webhook/admin/properties',
    {
      method: 'POST',
      headers: {
        'Content-Type': 'application/json',
        'Authorization': `Bearer ${token}`
      },
      body: JSON.stringify({
        action: 'update',
        user: '錦宣',
        id: propertyId,
        data: updates
      })
    }
  );

  return await response.json();
}
\end{lstlisting}

\newpage

\section{附錄:資料表欄位結構}

\subsection{資料查詢端點}

\begin{tabularx}{\textwidth}{|l|X|}
\hline
\textbf{項目} & \textbf{說明} \\
\hline
API 路徑 & /webhook/all-data \\
\hline
請求方法 & GET \\
\hline
需要認證 & 否 \\
\hline
說明 & 取得所有資料(物件、社區、使用者) \\
\hline
\end{tabularx}

\subsection{物件資料表(properties\_for\_sale)}

從 API 端點 \texttt{/webhook/all-data} 取得的實際欄位結構:

\begin{tabularx}{\textwidth}{|l|l|l|X|}
\hline
\textbf{欄位名稱} & \textbf{類型} & \textbf{可空} & \textbf{說明} \\
\hline
id & UUID & 否 & 唯一識別碼 \\
\hline
community\_name & string & 否 & 社區名稱 \\
\hline
total\_price & number & 是 & 總價(萬元) \\
\hline
total\_ping & number & 是 & 總坪數 \\
\hline
parking\_ping & number & 是 & 車位坪數 \\
\hline
parking\_price & number & 是 & 車位價格(萬元) \\
\hline
notes & string & 是 & 備註 \\
\hline
floor\_info & string & 是 & 樓層資訊(如:15樓 / 共15樓) \\
\hline
address & string & 是 & 地址 \\
\hline
status & string & 是 & 狀態(專任/一般/下架/屋主等) \\
\hline
layout & string & 是 & 格局(如:3房2廳2衛) \\
\hline
agent & string & 否 & 經辦人(建立者) \\
\hline
maintainer & string & 否 & 維護人(最後修改者) \\
\hline
created\_at\_source & date & 否 & 建立日期 \\
\hline
updated\_at\_source & date & 否 & 更新日期 \\
\hline
imported\_at & timestamp & 否 & 系統匯入時間 \\
\hline
photo\_paths & array & 是 & 照片路徑陣列 \\
\hline
cover\_photo\_path & string & 是 & 封面照片路徑 \\
\hline
\end{tabularx}

\subsection{社區資料表(communities)}

\begin{tabularx}{\textwidth}{|l|l|l|X|}
\hline
\textbf{欄位名稱} & \textbf{類型} & \textbf{可空} & \textbf{說明} \\
\hline
id & UUID & 否 & 唯一識別碼 \\
\hline
builder & string & 否 & 建商名稱 \\
\hline
community\_name & string & 否 & 社區名稱 \\
\hline
completion\_date & string & 是 & 完工年份(民國年) \\
\hline
total\_units & string & 是 & 總戶數 \\
\hline
unit\_area\_range & string & 是 & 坪數範圍(格式:最小/最大) \\
\hline
agent & string & 否 & 經辦人(建立者) \\
\hline
maintainer & string & 否 & 維護人(Creator 或最後修改者) \\
\hline
created\_at\_source & date & 否 & 建立日期 \\
\hline
updated\_at\_source & date & 否 & 更新日期 \\
\hline
\end{tabularx}

\subsection{使用者資料表(users)}

\begin{tabularx}{\textwidth}{|l|l|l|X|}
\hline
\textbf{欄位名稱} & \textbf{類型} & \textbf{可空} & \textbf{說明} \\
\hline
id & UUID & 否 & 唯一識別碼 \\
\hline
email & string & 否 & 電子郵件(唯一) \\
\hline
username & string & 否 & 使用者帳號 \\
\hline
name & string & 否 & 使用者姓名 \\
\hline
display\_name & string & 是 & 顯示名稱 \\
\hline
title & string & 是 & 職稱 \\
\hline
role & string & 否 & 角色(user/manager/admin) \\
\hline
Creator & string & 是 & 建立者 \\
\hline
maintainer & string & 是 & 維護人 \\
\hline
created\_at\_source & date & 是 & 建立日期 \\
\hline
updated\_at\_source & date & 是 & 更新日期 \\
\hline
\end{tabularx}

\subsection{API 回應範例}

\textbf{GET /webhook/all-data 回應格式:}

\begin{lstlisting}[language=JavaScript]
{
  "properties_for_sale": [
    {
      "id": "82f8542a-63d2-46b6-bfdc-71480d2e702e",
      "community_name": "新潤A18",
      "total_price": 1028,
      "total_ping": 13.764,
      "parking_ping": null,
      "parking_price": null,
      "notes": null,
      "floor_info": "15樓 / 共15樓",
      "address": "桃園市中壢區春德路",
      "status": "一般",
      "layout": null,
      "agent": "錦宣",
      "maintainer": "錦宣",
      "created_at_source": "2025-12-21",
      "updated_at_source": "2025-12-21",
      "imported_at": "2025-12-21T13:46:04.601305+00:00",
      "photo_paths": [],
      "cover_photo_path": null
    }
    // ... 更多物件
  ],
  "communities": [
    // 社區資料
  ],
  "users": [
    // 使用者資料
  ]
}
\end{lstlisting}

\end{document}
